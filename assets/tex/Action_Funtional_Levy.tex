\documentclass[10pt,reqno]{amsart}
\usepackage{graphicx}
   % MODIFYING AMSART.CLS:
     \makeatletter
     \def\section{\@startsection{section}{1}%
     \z@{.7\linespacing\@plus\linespacing}{.5\linespacing}%
     {\bfseries%\normalfont\scshape
     \centering
     }}
     \def\@secnumfont{\bfseries}
     \makeatother
   % END OF MODIFICATION OF AMSART.CLS.
\setlength{\textheight}{19.5 cm}
\setlength{\textwidth}{12.5 cm}
\newtheorem{theorem}{Theorem}[section]
\newtheorem{lemma}[theorem]{Lemma}
\newtheorem{proposition}[theorem]{Proposition}
\newtheorem{corollary}[theorem]{Corollary}
\theoremstyle{definition}
\newtheorem{definition}[theorem]{Definition}
\newtheorem{example}[theorem]{Example}
\theoremstyle{remark}
\newtheorem{remark}[theorem]{Remark}
\numberwithin{equation}{section}
\setcounter{page}{1}
%%%%% End of setup (Do not change) %%%%%%%%%%%%%%%%%%%%%%%%%%%%%%%%%%%%%%%%%%%%

\begin{document}


\begin{theorem}[Contraction Principle]
Let $(M_{1},d_{1})$, $(M_{2},d_{2})$ be metric spaces and $f: M_{1}\rightarrow M_{2}$ be a continuous function. Suppose that a family $(\mu_{\varepsilon})_{\varepsilon>0}$ of probability measures on $M_{1}$ satisfies a large deviation principle with action functional $I$.
Then the sequence of image measures $(\nu_{\varepsilon})_{\varepsilon>0}$ defined by $\nu_{\varepsilon}:=\mu_{\varepsilon}\circ f^{-1}$ on $M_{2}$, obeys a large deviation principle with action functional
\begin{equation*}
S(y):=\inf\{I(x): x\in M_{1}, y=f(x)\}.
\end{equation*}
\end{theorem}
\begin{proof}
Since $I$ is lower semicontinuous, it attains its minimum on compact sets. This implies that for any $y\in M_{2}$ and $S(y)<\infty$,
there exists $x\in M_{1}$ such that $f(x)=y$ and $S(y)=I(x)$. Then
\begin{equation*}
\Phi_{S}(r)=\{y\in M_{2}; S(y)\leq r\}=f(\Phi_{I}(r))~~~~\text{for}~~~~r\geq0.
\end{equation*}
In particular, $\Phi_{S}(r)$ is compact, i.e., $S$ is an action functional. Now let $U$ be an open set in $M_{1}$. Since $f$ is continuous, we know
$f^{-1}(U)$ is open. Apply the large deviation lower bound to $f^{-1}(U)$  and obtain
\begin{equation*}
\liminf_{\varepsilon\rightarrow0}\varepsilon\log\nu_{\varepsilon}(U)=\liminf_{\varepsilon\rightarrow0}\varepsilon\log\mu_{\varepsilon}(f^{-1}(U))\geq-\inf_{x\in f^{-1}(U)}I(x)=-\inf_{y\in U}S(y).
\end{equation*}
When $F$ is a closed set in $M_{1}$, the upper bound
\begin{equation*}
\limsup_{\varepsilon\rightarrow0}\varepsilon\log\nu_{\varepsilon}(F)=\limsup_{\varepsilon\rightarrow0}\varepsilon\log\mu_{\varepsilon}(f^{-1}(F))\leq-\inf_{x\in f^{-1}(F)}I(x)=-\inf_{y\in F}S(y)
\end{equation*}
follows in the same way.
\end{proof}


\end{document}
